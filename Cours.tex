%%%%%%%%%%%%%
% Pr�ambule %
%%%%%%%%%%%%%

\documentclass[12pt,a4paper,oneside,fleqn]{book} % Utiliser article si n�cessaire
\usepackage[latin1]{inputenc} % Pour les accents
\usepackage[T1]{fontenc} 
\usepackage[francais]{babel} % Pour la typographie
\usepackage[usenames]{color} % Pour les couleurs
\usepackage{amssymb} % Pour les symboles
\usepackage{amsmath} % Pour les maths
\usepackage{amsthm} % Pour les th�or�mes
\usepackage{fullpage} % Pour les marges
\usepackage{setspace} % Pour les espaces
\usepackage{array} % Pour les tableaux
\usepackage{tikz} % Pour les graphes
\usepackage{pgfplots} % Pour les fonctions

\setcounter{secnumdepth}{3}
\setcounter{tocdepth}{2}
\renewcommand\thesection{\Roman{section}}
\renewcommand\thesubsection{\quad\arabic{subsection}}
\renewcommand\thesubsubsection{\qquad\alph{subsubsection}}
\newtheorem{example}{Exemple}
\newtheorem{remark}{Remarque}

% Commandes ponctuelles

\newcommand{\C}{\mathbb{C}}
\newcommand{\R}{\mathbb{R}}
\newcommand{\diff}[1]{\,\mathrm{d}#1}
\renewcommand{\L}[1]{\mathcal{L}\left\{#1\right\}}
\newcommand{\Z}[1]{\mathcal{Z}\left\{#1\right\}}
\newcommand{\Li}[1]{\mathcal{L}^{-1}\left\{#1\right\}}
\newcommand{\Zi}[1]{\mathcal{Z}^{-1}\left\{#1\right\}}
\onehalfspacing
\onehalfspacing

%%%%%%%%%%%%
% Document %
%%%%%%%%%%%%

\begin{document}

\title{Deuxi�me ann�e --- M�thodes math�matiques}
\author{Richard \textsc{Degenne}, L3-B}
\date{\today}


\maketitle

\pagebreak

\tableofcontents

\pagebreak

\chapter{Int�grales g�n�rales, ou impropres}

\section{D�finition}

Soit $I\subset\R$ un intervalle dont les extr�mit�s $a<b$ (pouvant �tre $\pm\infty$) sont exclues et $f:I\to\R$ une fonction continue par morceaux sur $I$. Pour chaque intervalle $[x;y]\in I$, l'int�grale $\int_x^y f(t)\diff{t}$ est bien d�finie.

Si, pour $x$ tendant vers $a$ et $y$ tendant vers $b$, l'int�grale admet une limite finie, alors on dit que l'int�grale impropre $\int_a^b f(t)\diff{t}$ est convergente, et, par d�finition,
\[\int_a^b f(t)\diff{t}=\lim_{x\to a,y\to b}\int_x^y f(t)\diff{t}\]

Si la limite est infinie ou s'il n'y a pas de limite, on dit que l'int�grale impropre est divergente.

\subsection{Relation de Chasles}

Pour $c\in I$, on a
\[\int_x^y f(t)\diff{t} = \int_x^c f(t)\diff{t} \int_c^y f(t)\diff{t}\]

Si une int�grale impropre converge, alors toute relation de Chasles form� � partir de cette int�grale converge �galement. En prenant le cas particulier $c=a$, on peut en d�duire que
\[\int_a^b f(t)\diff{t}\text{ converge}\begin{array}{l}\iff\lim_{y\to b}\int_a^y f(t)\diff{t}\text{ existe et est fini.}\\\iff\lim_{x\to a}\int_x^b f(t)\diff{t}\text{ existe et est fini.}\end{array}\]

\section{Exemples}

\begin{example}
	\begin{align*}
		\int_1^x\frac{1}{t^2}\diff{t} & = \left[-\frac{1}{t}\right]_1^x\\
		                              & = 1-\frac{1}{x}\\
	\end{align*}
\end{example}

\begin{example}
	\begin{align*}
		\int_1^\infty\frac{1}{t^2}\diff{t} & = \lim_{x\to+\infty}\int_1^x\frac{1}{t^2}\diff{t}\\
		                                   & = \lim_{x\to+\infty}1-\frac{1}{x}\\
		                                   & = 1
	\end{align*}
\end{example}

\begin{example}
	\begin{align*}
		\int_1^\infty\frac{1}{t}\diff{t} & = \lim_{x\to+\infty}\int_1^x\frac{1}{t}\diff{t}\\
		                                 & = \lim_{x\to+\infty}1-\ln(x)\\
		                                 & = +\infty
	\end{align*}
\end{example}

\begin{example}
	\begin{align*}
		\int_0^x\cos(t)\diff{t} & = [\sin(t)]_{0}^{x}\\
		                        & = \sin(x)
	\end{align*}

	Or, $\lim_{x\to\infty}\sin(x)$ n'existe pas. Donc, $\int_0^x \cos(t)\diff{t}$ diverge.
\end{example}

\begin{example}
	\begin{align*}
		\int_0^\infty e^{-t}\diff{t} & = \lim_{x\to\infty}\int_0^x e^{-t}\diff{t}\\
		                             & = \lim_{x\to\infty}[-e^{-t}]_0^x\\
					     & = \lim_{x\to\infty}1-e^{-x}\\
					     & = 1
	\end{align*}
\end{example}

\begin{example}
	\begin{align*}
		\int_0^1 \frac{1}{\sqrt{t}}\diff{t} & = \lim_{x\to 0}\int_x^1 \frac{1}{\sqrt{t}}\diff{t}\\
		                                    & = \lim_{x\to 0}\left[2\sqrt{t}\right]_x^1\\
					            & = 2
	\end{align*}
\end{example}

\section{Propri�t�s}

Si $a$ et $b$ sont des valeurs finies et si $f$ est une fonction continue dans l'un des intervalles $[a;b[$, $]a;b]$ ou $]a;b[$ qui se prolonge par continuit� � $[a;b]$, alors $\int_a^b f(t)\diff{t}$ converge.

\begin{example}
\[\int_0^1\frac{\sin(t)}{t}\diff{t}\]
est convergente.\footnote{\emph{cf.} le d�veloppement limit� de $\sin(t)$ au voisinage de $t$.}
\end{example}

�videmment, les r�sultats et propri�t�s valables pour les int�grales classiques (changements de variable, int�gration par parties, lin�arit�,\ldots) restent valables pour les int�grales impropres.

\section{Int�grale absolument convergente}

Soit $I$ un intervalle d'extr�mit�s $a$ et $b$ appartenant � $\bar\R = \R\backslash\pm\infty$. Soit $f:I\to\R$ une fonction continue par morceaux et $g:I\to\R^+$ telle que $\forall t\in I,\lvert f(t)\rvert \le g(t)$. On a alors
\[\int_a^b g(t)\diff{t}\text{ converge}\implies \int_a^b f(t)\diff{t}\text{ converge}\]

Ainsi, on dit que $\int_a^b f(t)\diff{t}$ est absolument convergente si $\int_a^b\lvert f(t)\rvert\diff{t}$ converge. Toute int�grale imporpre absolument convergente est convergente.

\begin{remark}
	La r�ciproque est g�n�ralement fausse. Par exemple, $\int_0^\infty\frac{\sin(t)}{t}\diff{t}$ converge, mais $\int_0^\infty\left\lvert\frac{\sin(t)}{t}\right\rvert\diff{t}$ diverge.
\end{remark}
\end{document}
